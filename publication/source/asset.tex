% Ce fichier permet de définir:
%   - Les styles
%   - Les commandes
%   - Les environnements
% Utilisés pour mettre en forme les livres.

% ---------------------------------------------------------------------
%   Définitions de constantes
% ---------------------------------------------------------------------
\newcommand{\COLLECTION}{\emph{COLLECTION NARRATION}} % pour \COLLECTION{}

% ---------------------------------------------------------------------
%   Font
% ---------------------------------------------------------------------
% Je change la police par défaut pour :
\usepackage{times}

% ---------------------------------------------------------------------
%   Définition des couleurs utilisées
% ---------------------------------------------------------------------
\definecolor{fondexemple}{rgb}{0.98,0.98,0.99}
% \definecolor{fondexemple}{cmyk}{6,5,0,0}
\definecolor{bordexemple}{rgb}{0.7,0.7,0.7}

% ---------------------------------------------------------------------
%   Commandes pour les styles
% ---------------------------------------------------------------------

% Quand un mot est <tt>...</tt> (terme technique non indexé)
\newcommand{\tterm}[1]{\texttt{\textbf{#1}}}

% ---------------------------------------------------------------------
%   Environnements
% ---------------------------------------------------------------------

% === Main Environnement ===
% L'environnement que doit appeler tout environnement pour un
% document d'écriture en exemple, même de type scénario.
% Cf. les environnements généraux définis ci-dessous.
% 
\newcommand{\docAuteur}[1]{%
  \noindent
  \setlength{\parindent}{0em}%
  \vspace{12pt}\par
  \fcolorbox{bordexemple}{fondexemple}{%
    \begin{minipage}{1\textwidth}
      \vspace{12pt}%
      #1%
      \vspace{12pt}%
    \end{minipage}%
  }
  \vspace{8pt}\par%
  % Remettre la config initiale
  % ---------------------------
  % TODO: Plus tard, il faudra reprendre les définitions par défaut
  % EN fait : faire une commande pour remettre les valeurs par défaut.
  \setlength{\parindent}{2em}%
  \fontfamily{}
  \selectfont
}

% --- Environnement Scénario ---
% Commande sous classe de \docAuteur
% Elle modifie simplement la font et est utilisé
% Par tout environnement scénario
\newcommand{\docScenario}[1]{%
  \fontfamily{phv}
  \selectfont % pour que la font soit prise en compte
  \docAuteur{#1}
}

% --- Environnement Synopsis ---
\newcommand{\docSynopsis}[1]{#1}

% ---------------------------------------------------------------------
%   Méthodes générales
% ---------------------------------------------------------------------

% Sous-environnement permettant de définir les 
% décalage de texte
% Notes
%   * Usage : \OffText{<marge gauche>}{<marge droite>}
%   * On peut définir les polices dans l'environnement principal
%     cf. \docScenario ci-dessus
% 
\newenvironment{OffText}[2]{%
\begin{list}{}{%
\setlength{\leftmargin}{#1}%
\setlength{\rightmargin}{#2}%
}
\setlength{\parskip}{0pt}%
\setlength{\parindent}{2em}%
\scriptsize\item[]%
}{%
\end{list}
}

% ---------------------------------------------------------------------
%   Pour les scénarios
% 
%     \scenarioIntitule{....}
%     \scenarioAction{....}
%     \scenarioPersonnage{...}
%     \scenarioNoteJeu{...}
%     \scenarioDialogue{...}
% 
% Notes
%   * Ces commandes sont incluses dans l'environnement \docScenario
%     cf. ci-dessus
% ---------------------------------------------------------------------
\newcommand{\scenarioScene}[1]{%
  \begin{OffText}{12pt}{12pt}
    {\bfseries #1}
  \end{OffText}%
}
\newcommand{\scenarioCharacter}[1]{%
  \begin{OffText}{4cm}{12pt}
    {\bfseries #1}
  \end{OffText}%
}
\newcommand{\scenarioNoteJeu}[1]{%
  \begin{OffText}{4cm}{3cm}
    \vspace{-6pt}
    {\itshape #1}
  \end{OffText}%
}
\newcommand{\scenarioDialog}[1]{%
  \begin{OffText}{2.5cm}{2cm}
    \begin{flushleft}
      \vspace{-6pt}
      #1
    \end{flushleft}
  \end{OffText}%
}
\newcommand{\scenarioAction}[1]{%
  \begin{OffText}{22pt}{24pt}
    #1
  \end{OffText}%
}  