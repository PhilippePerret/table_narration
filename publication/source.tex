\documentclass[11pt,french,twoside,a5paper]{book}

%=============== Entête ==================
\usepackage[french]{babel}
\usepackage[a5paper]{geometry}
\usepackage[utf8]{inputenc}
\usepackage[T1]{fontenc}
\usepackage[cyr]{aeguill}
\usepackage{epsfig}
\usepackage{float}
\usepackage{color}
\usepackage{makeidx}
\usepackage[xetex, pagebackref]{hyperref}
\title{Le vrai titre du livre}
\author{Philippe Perret}
\makeindex
\newcommand{\mapolicebackref}[1]{
    \hspace*{\fill} \mbox{\textit {\small #1}}
}
%pour gérer les backref dans la bibliographie
\renewcommand*{\backref}[1]{}
\renewcommand*{\backrefalt}[4]{%
\ifcase #1 \mapolicebackref{pas de citations}
    \or \mapolicebackref{page #2}
    \else \mapolicebackref{#1 pages #2}
\fi
}
% Séparateurs :
\renewcommand*{\backreftwosep}{, }
\renewcommand*{\backreflastsep}{, }

%=============== /Entête ==================

\begin{document}
\maketitle

\tableofcontents
\chapter{Introduction}
\section{La fonction des documents}
Deux fonctions à bien distinguer\newline
[Sur le fait qu\'il faut bien distinguer la fonction de vente et la fonction de travail]\newline
Document comme outil de vente\newline
Document comme outil de travail\newline
\section{La notion d\'échelle}
\chapter{Les documents}
\section{Le pitch}
\section{Le synopsis}
\section{Le Traitement}

\renewcommand{\bibname}{Filmographie}
\begin{thebibliography}{1}
\bibitem{Metropolis}
  \textsc{\textbf{Métropolis (Metropolis)}},
  Thea Von Harbou (roman et script), Fritz Lang (script), Fritz Lang,
  1927.\newline
  {\advance\baselineskip -3pt {\scriptsize Un monde séparé en deux~: d\'un côté les “penseurs”, qui pensent mais ne connaissent rien au fonctionnement des choses, de l\'autre les “travailleurs”, qui travaillent et produisent mais n\'ont aucune vision, aucun projet. Un jour, un “penseur” descend dans les bas-fonds, là où vivent les “travailleurs”.} \par}\newline
  \backref
\end{thebibliography}
\printindex
\end{document}
